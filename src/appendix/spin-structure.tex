\section{Spin structure}
\label{sec:spin-structure}

In the notes, we focused primarily on bosonic Chern-Simons theories. Fermionic CS theories require the choice of a spin structure. Here, I will review spin structures.

\note{To do: check out \cite[Sec.~46.7]{grensing2013structural-aspe} and see if there's anything more to add here.}


\subsection{\texorpdfstring{\( k \)}{k} can be odd of spin Chern-Simons}

Let's start by restricting to \( \U(1) \) CS theories. From \cref{sec:chern-classes}, \( F / 2\pi \in H^2_{DR}(M, \bbR) \).
Since \( \bbR \) is a field, \( H^\ast(M, \bbR) \cong \Hom(H_\ast(M), \bbR) \).
By de Rham's theorem, \( H^\ast_{DR}(M, \bbR) \cong H^\ast(M, \bbR) \cong \Hom(H_\ast(M), \bbR) \) via \( \omega \in H^\ast_{DR}(M, \bbR) \) maps to \( (H_\ast(M) \ni c \mapsto \int_c \omega \in \bbR) \in \Hom(H_\ast(M), \bbR) \).
Thus, we see that \( F / 2\pi \in \Hom(H_2(M), \bbZ) \), and thus \( F \in H^2(M, \bbZ) \)\footnote{\url{https://math.stackexchange.com/questions/399507/how-do-i-know-when-a-form-represents-an-integral-cohomology-class}}.

Since \( F/2\pi \in H^2(M, \bbZ) \), we know\footnote{\url{https://math.stackexchange.com/q/29797/395731}} that the cup product with itself is an element of \( H^4(M, \bbZ) \), and hence
\begin{equation}
    \frac{k}{8\pi^2}\int F \wedge F = \frac{k}{2}I
\end{equation}
where \( I \in \bbZ \).
if \( k \) is even, then the whole thing is thus an integer with or without a spin structure.
If \( k \) is odd, then the CS theory is only well-defined if \( I \in 2\bbZ \).
A spin four-manifold must have even intersection form\footnote{\url{https://en.wikipedia.org/wiki/Intersection_form_of_a_4-manifold}}, and the converse. Thus, if \( M \) is spin, then \( \int (F/2\pi)\wedge (F/2\pi) = I \in 2\bbZ \) so that the CS is well-defined, and the converse.
\note{To do: understand Wu's formula via Stiefel-Whitney classes \cite[Thm.~11.7]{nakahara2003geometry-topolo}}
Thus, we now understand why spin CS theories work for arbitrary integer \( k \), while bosonic CS theories only work when \( k \) is even.


There is another way to prove that \( \frac{1}{4\pi^2}\int_N F \wedge F \in 2\bbZ \) for a closed manifold \( N \) using the index theorem for the Dirac operator \cite[pp.~412]{dijkgraaf1990topological-gau} \note{sounds cool, work thorugh this!}.



\subsection{When we want Spin vs \texorpdfstring{\( \text{Spin}^\bbC \)}{SpinC}}

\emph{From \cite{avis1980generalized-spi}}

(I will be a bit sloppy going back and forth talking about principle bundles and their associated vector bundles; this does not really matter since by definition their transition functions are the same)

When we have a Riemannian manifold \( (M, g) \) with \( n=\dim M \), the transition functions lie in \( \O(n) \).
If we can reduce the transition functions to lie in \( \SO(n) \), then \( M \) is orientable.
Many objects we deal with transforms in a representation of \( \SO(n) \) and is thus acted on by these transition functions accordingly.
However, fermions transform in a projective represtation of \( \SO(n) \), which corresponds to a representation of the double cover \( \operatorname{Spin}(n) \).
Thus, we need a principle \( \operatorname{Spin}(n) \)-bundle. We construct such a bundle by lifting the transition functions from the original \( \SO(n) \)-bundle. Spinors are then sections of the associated vector bundle.
Refer to Ref.~\cite[Ch.~11]{nakahara2003geometry-topolo} for how we eventually find that the first Stiefel-Whitney class tells us whether the manifold is orientable and the second Stiefel-Whitney class tells us if it is spin.

On the other hand, sometimes requiring a Spin bundle is unnecessarily restrictive. In particular, suppose that we have a fermion that is charged under a \( G \)-gauge field. Let \( \tilde G \) be a double cover, \( G = \tilde G / \bbZ_2 \). We could take a \( G\times \SO(n) \)-bundle and try to lift it to a \( G\times \Spin(n) \)-bundle, in which case we run into the same analysis as before.
There are cases, though, where the second SW class is nonzero (and so we cannot do the lifting) and yet we can still define fermions!
We assume that the center \( \bbZ_2 \) factors of \( \tilde G \) and \( \Spin(n) \) act the same on the fermions.
In this case, the group that couples to the fermions is actually \( (\tilde G \times \Spin(n)) / \bbZ_2 \).
So we actually only need a \( \Spin^G(n) \)-bundle.

So we care about \( \Spin^G(n) = \Spin(n) \times_{\bbZ_2} G \), where we have the \( \bbZ_2 = \{e, a\} \) equivalence \( (u, g) \sim (-u, ag) \) for \( u\in\Spin(n), g\in G \). For some reason, \( \Spin^\bbC \coloneqq \Spin^{\U(1)} \).


\subsubsection*{Obstructions}

In the case of \( \Spin \) bundles, the second SW class was an obstruction.
In the case of \( \Spin^G \) bundles, it turns out that the third \emph{integral} SW class is an obstruction. Let's show the following (from Wikipedia).

\begin{proposition}
    A \( \Spin^\bbC \) structure exists on \( M \) iff \( M \) is orientable and its second SW class (an element of \( H^2(M, \bbZ_2) \)) is in the image of \( H^2(M, \bbZ) \to H^2(M, \bbZ_2) \). Equivalently, this is iff the third integral SW class (an element of \( H^3(M, \bbZ) \) vanishes).
\end{proposition}
\begin{proof}[Proof sketch]
    Using the short exact sequence \( 0 \rightarrow \bbZ \xrightarrow{\times 2} \bbZ \to \bbZ_2 \to 0 \), we get the long exact sequence in cohomology (see my other notes on the Ext functor, etc)
    \begin{equation}
        \dots \to H^2(M, \bbZ)  \xrightarrow{\times 2} H^2(M, \bbZ) \to H^2(M, \bbZ_2) \xrightarrow{\beta} H^3(M, \bbZ) \to \dots
    \end{equation}
    Apparently \( \beta \) is something called the Bockstein homomorphism.
    Clearly the two iff's are equivalent by exactness; ie.~by exactness, the image of the second SW class is in \( H^2(M, \bbZ) \to H^2(M, \bbZ_2) \) iff it is in the kernel of \( \beta \).
    It turns out that the third integral SW class is equal to \( \beta \) on the second SW class (not sure why this is exactly, I suppose I need to learn more about the Bockstein homomorphism).

    Now we show the first iff. Suppose that the second SW class \( w_2 \in H^2(M, \bbZ_2) \) is nonzero. Suppose that we have a \( \Spin \) ``bundle'' (not actually a bundle because \( w_2 \neq 0 \) so it fails the triple overlap condition, so I'll just keep quotes around it).
    The failure of a lift comes from the triple overlap conditions with the minus signs coming from \( \bbZ_2 \).
    So to cancel this issue, we tensor our \( \Spin \) ``bundle'' with a \( \U(1) \) ``bundle'' that has the same \( w_2 \).

    A true \( U(1) \) bundle is classified by its Chern class in \( H^2(M, \bbZ) \). Going through the sequence, we double the class. So legitamate bundles correspond to even elements in the second \( H^2(M, \bbZ) \).
    By exactness, legitamate bundles correspond to zero elements in the image in \( H^2(M, \bbZ_2) \).

    The \( \U(1) \) ``bundle'' will cancel the obstruction if the image is \( w_2 \). Thus we see that a \( \Spin^\bbC \) bundle exists iff \( w_2 \) is in the image of the natural map \( H^2(M, \bbZ) \to H^2(M, \bbZ_2) \).
\end{proof}

Apparently every \( 4 \)-manifold has a \( \Spin^\bbC \) structure.



\subsection{How does the spin structure affect the theory}

\cite[Sec.~2.5,~2.6]{moore2019introduction-to}, \cite{belov2005classification-}

We saw above that a \( \Spin^\bbC \) bundle exists on a four manifold \( M \) iff the second SW class \( w_2 \in H^2(M, \bbZ_2) \) is in the image of the map \( H^2(M, \bbZ) \to H^2(M, \bbZ_2) \).
In other words, we need an integral lift \( \hat w_2 \in H^2(M, \bbZ) \) of \( w_2 \in H^2(M, \bbZ_2) \).
We define a spin structure on \( M \) by \( \hat w_2 \).
Since \( \hat w_2, F/2\pi \in H^2(M, \bbZ) \), their cup product (ie.~wedge product) is in \( H^4(M, \bbZ) \).

In particular, apparently \( \hat w_2 \) is always a characteristic vector, meaning that \( \int F/2\pi \wedge w_2 = \int F/2\pi \wedge F/2\pi \pmod 2 \) \note{why is this?}.
Thus, we can define the action \( \frac{k}{8\pi^2}\int_M F \wedge (F \wedge \hat w_2) \) and it is well-defined for all \( k\in \bbZ \).
\note{I'm a little bit confused. In the other notes that I found and as discussed above, they say that \( \int F/2\pi \wedge F/2\pi \in 2\bbZ \), and so the CS theory is immediately well-defined as long as the manifold is spin (ie.~independent of spin structure). But here, it seems like he does not assume this to be the case, but instead uses that \( \hat w_2 \) is a characteristic vector to show that the integral is even.}

\note{In particular, given the discussion above coming from \cite{dijkgraaf1990topological-gau}, it seems like the braiding for a spin CS theory described by a \( K \) matrix would still be given entirely by \( K^{-1} \), and the spin structure doesn't change anything. On the other hand, given the description here in terms of \( \hat w_2 \), it seems like \( \hat w_2 \) might affect the braiding. So what's happening? In particular, from \cite[Eq.~2.355]{moore2019introduction-to}, it seems like the braiding will be discribed by \( (K+\text{something})^{-1}  \).}


\note{\cite{belov2005classification-} is better than \cite{moore2019introduction-to}. It seems to match \cite{dijkgraaf1990topological-gau}. But it also has a section about how it depends on spin structure. Need to understand this better. Need to understand how they get to Eq.~2.9. Then, what does Eq.~2.9 imply for braiding. I think it's not supposed to affect anything because it seems equivalent to shifting \( A \to A + \epsilon \), where \( \dd\epsilon=0 \) and \( \int \epsilon \in \{0,1\} \). So it seems like it is just adjusting the holonomies. But in the quantum theory, we integrate over all \( A \), so this shouldn't affect anything. But I think maybe it matters when there is a boundary?}

\note{I think spin structure does not affect the K matrix stuff, and I think it only affects things when we couple to other things. From \cite{belov2005classification-}, it looks like a change in spin structure can be equivalently thought of as \( A \to A + f \). But since the path integral is over all \( A \), this should not affect things in the pure gauge theory.}

